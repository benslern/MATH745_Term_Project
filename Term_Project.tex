\documentclass[12pt]{article}


\usepackage{xcolor}
\usepackage{graphicx}
\usepackage{amsmath}
\usepackage{amsthm}
\usepackage{amssymb}
\usepackage{amsfonts}

\pagecolor[rgb]{0.2,0.2,0.2}
\color[rgb]{0.95,0.95,0.95}

\usepackage{times}\usepackage{setspace}
\doublespacing

\usepackage{hyperref}
\hypersetup{
 colorlinks,
 linkcolor={blue!90!black},
 citecolor={red!80!black},
 urlcolor={blue!50!black}
}
\urlstyle{same}
\pagenumbering{arabic}

\usepackage[letterpaper, total={6.25in, 8.5in},top=35mm]{geometry}
\usepackage{fancyhdr}
\pagestyle{fancy}
\lhead{Math 745 Term Project - Noah Bensler}
\rhead{McMaster University - CSE}
\renewcommand{\headrulewidth}{0.4pt}
\renewcommand{\footrulewidth}{0.4pt}


\newcommand{\bds}{\boldsymbol}
\newcommand{\la}{\left\langle}
\newcommand{\ra}{\right\rangle}
\newcommand{\lb}{\left\lbrace}
\newcommand{\rb}{\right\rbrace}

\def\w{\omega}
\def\wt{\omega_t}
\def\wx{\omega_x}
\def\wxx{\omega_{xx}}
\def\L{\mathcal{L}}
\def\P{\mathbf{P}}

\DeclareMathOperator*{\argmax}{argmax}

\newtheorem{thm}{Theorem}[section]

\numberwithin{equation}{section}

\begin{document}

\newgeometry{top=3.8cm,left=3.8cm,right=2.5cm,bottom=2.5cm}
\begin{titlepage}
   \begin{center}
       \vspace*{1cm}

	    ON A SEARCH FOR SINGULARITIES IN THE GENERALIZED CONSTANTIN–LAX–MAJDA EQUATION
	   
   \end{center}
\end{titlepage}

\setcounter{page}{0}
\begin{titlepage}
   \begin{center}
       \vspace*{3.8cm}

       ON A SEARCH FOR SINGULARITIES IN THE GENERALIZED CONSTANTIN–LAX–MAJDA EQUATION

       \vspace{2.5cm}

       By NOAH BENSLER, B.Sc.

       \vfill
            
       A Project Submitted to Math 745\\
            
       \vspace{0.8cm}
     
            
       Computational Science \& Engineering\\
       McMaster University\\
       ©Copyright by Noah Bensler, DATE
            
   \end{center}
\end{titlepage}
\restoregeometry

\pagenumbering{roman}
\setcounter{page}{2}

\pagebreak

\begin{flushleft}
\begin{large}
\textbf{Abstract}
\\~\\
\end{large}
\end{flushleft}
\noindent
\qquad For my term project I will compute and analyse solutions of the Generalized Constantine-Lax-Majda equation. The Constantine-Lax-Majda (CLM) equation was proposed as a one-dimensional model for the three-dimensional vorticity equation. The CLM equation was later expanded on by De Gregorio to include the convection term $u\omega_x$. The De Gregorio equation was generalized by Okamoto et al. to include the real parameter $a$ which determines the relative strength of the convection term. The result is the Generalized Constantin-Lax-Majda equation (GCLM). 
\\~\\
The CLM equation was proposed in 1985 and has been shown to exhibit finite time blow-up. While the De Gregorio equation was proposed shortly after in 1989, questions still remain as to whether or not the equation permits finite time blow-up. The goal of this analysis is to examine the affect the parameter $a$ has on the finite time blow-up of the GCLM equation on the periodic domain $x\in[-\pi,\pi]$. I will explore solutions of the GCLM for values of $a\in[-1,1]$. 
\\~\\
The analysis found RESULTS RESULTS RESULTS. CONCLUSION CONCLUSION CONCLUSION.

\pagebreak


{
  \hypersetup{linkcolor=white} %black
  \tableofcontents
}

\pagebreak

\pagenumbering{arabic}

\section{Introduction}

One of the 


\section{The Generalized Constantin–Lax–Majda Equation}

explore this by studying simpler one-dimensional models.

\subsection{The CLM Equation}

The Constantin-Lax-Majda equation was proposed as a one-dimensional model of the three-dimensional vorticity equation for an incompressible fluid \cite{constantin_simple_1985}.

\subsection{The De Gregorio Equation}

The De Gregorio equation was proposed as a one-dimensional model that improves on the CLM equation by ... \cite{de_gregorio_one-dimensional_1990}

\subsection{Generalization of the CLM Equation}
The De Gregorio equation was later generalized to the GCLM \cite{okamoto_generalization_2008}.

\section{Blow-up Criteria}
how do we define blow up

\section{Numerical Methods}
what numerical methods are used
\subsection{Domain and Grid}
a
\subsection{Time Stepping}
b
\subsection{Initial Conditions}
c

\section{Results}
these are the results

\section{Conclusion}
this is the conclusion

\pagebreak

\bibliographystyle{plain}
\bibliography{references}

\end{document}
