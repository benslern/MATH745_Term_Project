\documentclass[12pt]{article}


\usepackage{xcolor}
\usepackage{color}
\usepackage{graphicx}
\usepackage{amsmath}
\usepackage{amsthm}
\usepackage{amssymb}
\usepackage{amsfonts}
\usepackage[numbered,framed]{matlab-prettifier}
\usepackage[T1]{fontenc}
\usepackage{listings}
\lstset{
        basicstyle=\ttfamily,
        literate={~}{{\fontfamily{ptm}\selectfont \textasciitilde}}1
}

\pagecolor[rgb]{0.2,0.2,0.2}
\color[rgb]{0.95,0.95,0.95}

\newtheorem{thm}{Theorem}[section]

\usepackage{times}\usepackage{setspace}
\doublespacing

\usepackage{hyperref}
\hypersetup{
 colorlinks,
 linkcolor={blue!90!black},
 citecolor={red!80!black},
 urlcolor={blue!50!black}
}
\urlstyle{same}
\pagenumbering{arabic}

\usepackage[letterpaper, total={6.25in, 8.5in},top=35mm]{geometry}
\usepackage{fancyhdr}
\pagestyle{fancy}
\lhead{Math 745 Term Project --- Noah Bensler}
\rhead{McMaster University --- CSE}
\renewcommand{\headrulewidth}{0.4pt}
\renewcommand{\footrulewidth}{0.4pt}


\newcommand{\bds}{\boldsymbol}
\newcommand{\la}{\left\langle}
\newcommand{\ra}{\right\rangle}
\newcommand{\lb}{\left\lbrace}
\newcommand{\rb}{\right\rbrace}
\newcommand{\ls}{\left[}
\newcommand{\rs}{\right]}
\newcommand{\lr}{\left(}
\newcommand{\rr}{\right)}

\def\w{\omega}
\def\wt{\omega_t}
\def\wx{\omega_x}
\def\wxx{\omega_{xx}}
\def\wk{\hat{\omega}_k}
\def\wn{\omega^{(n)}}
\def\un{u^{(n)}}
\def\ux{u_x}
\def\ut{u_t}
\def\tn{t^{(n)}}
\def\L{\mathcal{L}}
\def\R{\mathbb{R}}
\def\sgn{\,\text{sgn}}

\DeclareMathOperator*{\argmax}{argmax}

\numberwithin{equation}{section}

\begin{document}

\newgeometry{top=3.8cm,left=3.8cm,right=2.5cm,bottom=2.5cm}
\begin{titlepage}
   \begin{center}
       \vspace*{1cm}

	    SINGULARITIES IN THE GENERALIZED CONSTANTIN–LAX–MAJDA EQUATION
	   
   \end{center}
\end{titlepage}

\setcounter{page}{0}
\begin{titlepage}
   \begin{center}
       \vspace*{3.8cm}

       ON A SEARCH FOR FINITE-TIME SINGULARITIES IN THE GENERALIZED CONSTANTIN–LAX–MAJDA EQUATION

       \vspace{2.5cm}

       By NOAH BENSLER, B.Sc.

       \vfill
            
       A Project Submitted to Math 745\\
            
       \vspace{0.8cm}
     
            
       Computational Science \& Engineering\\
       McMaster University\\
       ©Copyright by Noah Bensler, 17 December 2025.
            
   \end{center}
\end{titlepage}
\restoregeometry

\pagenumbering{roman}
\setcounter{page}{2}

\pagebreak

\begin{flushleft}
\begin{large}
\textbf{Project Proposal}
\end{large}
\end{flushleft}
\noindent
For my term project I will compute and analyse solutions of the Generalized Constantine-Lax-Majda equation. The Constantine-Lax-Majda (CLM) equation was proposed as a one-dimensional model for the three-dimensional vorticity equation and has the form\vspace{-12pt}
$$\wt - \ux\w = 0, \qquad \ux = H\w\vspace{-12pt}$$
where $\w$ is the vorticity of the fluid and $u$ is the fluid velocity. For this analysis the equations will be examined on the periodic domain $x\in[-\pi,\pi]$. The Hilbert transform of the vorticity is defined as
$$H\w(x,t) = \frac{1}{2\pi}\int_{-\pi}^{\pi} \w(y,t)\cot\lr\frac{x-y}{t}\rr \,dy,$$
which has a singularity at $x=y$ and so must be evaluated using the Cauchy Principle Value. The CLM equation was later expanded on by De Gregorio to include the convection term $u\wx$. The De Gregorio equation has the form\vspace{-12pt}
$$\wt + u\wx - \ux\w = 0, \qquad \ux = H\w.\vspace{-12pt}$$
The De Gregorio equation was later generalized by Okamoto et al. to include the real parameter $a$. The result is the Generalized Constantin-Lax-Majda equation (GCLM)\vspace{-12pt}
$$\wt + au\wx - \ux\w = 0, \qquad \ux = H\w.\vspace{-12pt}$$
The parameter $a$ determines the relative strength of the convection term. When $a=0$ the GCLM is equal to the CLM, and when $a=1$ it is equal to the De Gregorio equation. The CLM equation was proposed in 1985 and has been shown to exhibit finite time blow-up. While the De Gregorio equation was proposed shortly after in 1989, questions still remain as to whether or not the equation permits finite time blow-up. The goal of this analysis will be to examine the affect the parameter $a$ has on the finite time blow-up of the GCLM equation. I will explore solutions of the GCLM for values of $a\in[-1,1]$. 

\pagebreak

\begin{flushleft}
\begin{large}
\textbf{Abstract}
\\~\\
\end{large}
\end{flushleft}
\noindent
The Constantine-Lax-Majda (CLM) equation was proposed as a one-dimensional model for the three-dimensional vorticity equation. The CLM equation was later expanded on by De Gregorio to include the convection term $u\wx$. The De Gregorio equation was generalized by Okamoto et al. to include the real parameter $a$ which determines the relative strength of the convection term. The result is the Generalized Constantin-Lax-Majda equation (GCLM). In this analysis we examine the affect the parameter $a\in[-1,1]$ has on the finite time blow-up of the GCLM equation on the periodic domain $x\in[-\pi,\pi]$. 
\\~\\
The analysis found RESULTS RESULTS RESULTS. CONCLUSION CONCLUSION CONCLUSION.

\pagebreak


{
  \hypersetup{linkcolor=white} %black
  \tableofcontents
}

\pagebreak

\pagenumbering{arabic}

\section{Introduction}

The three-dimensional Euler equations are a simplified form of the Navier-Stokes equations which describe the flow of inviscid, incompressible fluids. The Euler equations are
\begin{subequations}\label{eq:Euler}
\begin{align}
\ut + u\cdot \nabla u + \nabla p &= 0, \\
\nabla \cdot u &= 0,
\end{align}
\end{subequations}
where $u$ is the velocity field and $p$ is the scalar pressure. By substituting in the vorticity ($\w = \nabla \times u$) these equations can be re-expressed in the three-dimensional vorticity equation
\begin{equation}\label{eq:Vorticity}
\wt + (u\nabla)\w = (\w\nabla)u,
\end{equation}
where the velocity can be computed from the vorticity field with the equation DOUBLE CHECK THIS
\begin{equation}\label{eq:Vel:Vort}
u = -\nabla \times (\nabla^{-1} \w).
\end{equation}
A central question in fluid dynamics is whether or not finite-time singularities can form in fluid flows with smooth initial conditions \cite{constantin_simple_1985}. As first demonstrated by Beale, Kato, and Majda, singularities can form if and only if the maximum vorticity of the flow becomes infinite \cite{beale_remarks_1984}. The vorticity equation allows for the direct modelling of the vorticity and so is a very useful tool for studying singularity formation in the Euler equations. WHY USE 1D MODELS.

\subsection{The CLM Equation}

The Constantin-Lax-Majda (CLM) equation was the first one-dimensional model of the three-dimensional vorticity equation for an incompressible fluid \cite{constantin_simple_1985}. The CLM equation is
\begin{equation}\label{eq:CLM}
\wt = H(\w)\w
\end{equation}
where $H$ is the Hilbert transform DOUBLE CHECK THIS
\begin{equation}\label{eq:Hilbert:Transform:R1}
H\w(x,t) = \frac{1}{2\pi}\int_{-\pi}^{\pi} \w(y,t)\cot\lr\frac{x-y}{2}\rr \,dy.
\end{equation}
The CLM equation was defined on the unbounded domain $R^1$ DOUBLE CHECK THIS. A significant advantage of this model was that its simplicity. This allowed Constantin et al. to prove the following Theorem \cite{constantin_simple_1985}.
\begin{thm}
Suppose $\w_0(x)$ is a smooth function decaying sufficiently rapidly as $|x|\rightarrow \infty$ ($\w_0 \in H^1(\R)$ suffices). Then the solution to the model vorticity equation in (\ref{eq:CLM}) is given by
\begin{equation}\label{eq:Explicit:Omega}
\w(x,t) = \frac{4\w_0(x)}{(2 - tH\w_0(x))^2 + t^2\w_0^2(x)}.
\end{equation}
\end{thm}
The CLM model is a very simple model. The biggest advantage it has is that it has an explicit solution which allows for ...\cite{constantin_simple_1985}. 

However, De Gregorio noted three shortcomings of the model \cite{de_gregorio_one-dimensional_1990}.
\begin{enumerate}
\item asdf
\end{enumerate}

\subsection{The De Gregorio Equation}

The De Gregorio equation was proposed as a one-dimensional model that improves on the CLM equation by ... \cite{de_gregorio_one-dimensional_1990}

\subsection{The Generalized CLM Equation}
The De Gregorio equation was later generalized to the GCLM \cite{okamoto_generalization_2008}.

\begin{equation}\label{eq:Hilbert:Transform}
H\omega(x,t) = \frac{1}{2\pi}\int_{-\pi}^{\pi} \omega(y,t)\cot\left(\frac{x-y}{2}\right) \,dy.
\end{equation}

\subsection{Blow-up Criteria}
Okamoto et al. (2008) prove Theorem 3.1 and Theorem 3.2.

\begin{thm}\label{thm:Theorem:3.1}
Let $a\in\R$ be given. For all $\omega_0\in H^1(S^1)/\R$, there exists a $T>0$ depending only on $a$ and $\|\omega_{0,x}\|$ such that there exists a unique solution $\omega\in C^0([0,T];H^1(S^1)/\R) \cap C^1([0,T];L^2(S^1)/\R)$ of (GCLM ref) with $\omega(0,x)=w_0$.
\end{thm}


\subsection{Overview}

In this paper we will conduct a search for singularity formation in the GCLM by 

\section{Numerical Methods}

We use the same numerical methods as \cite{okamoto_generalization_2008} which are described below. We use the notation $\hat{\omega}_k$ for the Fourier coefficients of the vorticity. There derivatives  The numerical integration techniques below describe the procedure to calculate the value of $\omega$ at step $n+1$ from the value at step number $n$. To distinguish the notation from the Fourier coefficient we use $\omega^{(n)}$, $u^{(n)}$, and $t^{(n)}$ to denote the vorticity, velocity, and time, respectively, at step $n$.


\subsection{Domain and Grid}

The vorticity $\omega($ is represented in physical space on a grid of $N = 2^14$ equidistant points on the domain $x\in[\pi,pi]$. Because the domain is periodic we must avoid double counting the grid point on the boundary. Therefore we set $x_0=-\pi$ and $x_N=\pi-h$, where $h$ is the uniform distance between points $x_i$ and $x_{i+1}$. The vorticity is represented in Fourier space with the truncated Fourier series
\begin{equation}\label{eq:Fourier}
\omega(t,x) = \sum_{k=-N/2}^{N/2 - 1}\hat{\omega}_k e^{ikx}.
\end{equation}

Transformation from physical to Fourier space, and vice versa, are performed using the standard Matlab fft and ifft functions. 

We compute the value of $u^{(n)}$ in Fourier Space using the formula
\begin{equation}
u^{(n)}(t,x) = 4\ln(2)\hat{\omega}^{(n)}_0 + \sum_{k=-N/2,k\neq 0}^{N/2 - 1}\frac{\hat{\omega}^{(n)}_k \exp(ikx)}{k}\,dv.
\end{equation}
 
 Compute $u_x$ in Fourier space. Compute $\omega_x$ in Fourier space. 
 
 The full derivation of each formula is shown in the Appendix. 

\subsection{Time Stepping}

When modelling time dependent PDEs it is best practice to use explicit time stepping methods for non-linear terms in order to reduce computation cost, and use implicit methods for linear term to improve stability \cite{protas_spectral_2025}. As described in prior sections, both the convection term and stretching term of the GCLM are non-linear. Therefore, as is used in \cite{okamoto_generalization_2008}  and \cite{hideki_numerical_2006}, we perform time stepping using the explicit fourth-order Runge-Kutta (RK4) method. The RK4 time stepping is performed with the formula
\begin{equation}\label{eq:RK4}
\omega^{(n+1)} = \omega^{(n)} + \frac{1}{6}(k_1 + 2k_2 + 2k_3 + k_4)
\end{equation}
where the $k_i$ terms are calculated using
\begin{subequations}
\begin{align}
k_1 &= f(\omega^{(n)},t^{(n)})  \\
k_2 &= f(\omega^{(n)}+ dt \cdot k_1/2,t^{(n)} + dt/2)  \\
k_3 &= f(\omega^{(n)}+ dt \cdot k_2/2,t^{(n)}+ dt/2)  \\
k_4 &= f(\omega^{(n)}+ dt \cdot k_3,t^{(n)}+dt) 
\end{align}
\end{subequations}
where $dt = 1E-4$ is the time step size. The function $f$ in the above equations is given by
\begin{equation}\label{eq:RHS}
f(\omega^{(n)},t^{(n)}) = \ux\omega - au\wx
\end{equation}

The derivatives $\ux$ and $\wx$ are computed in Fourier space. The

\subsection{Complete Numerical Method}


Combining the methods described above, we use the following procedure

\begin{enumerate}
  \item Convert $\omega^(n)$ from Physical space to Fourier space using FFT.
  \item Calculate $u_x$ 
\end{enumerate}


\subsection{Initial Conditions}

Okamoto et al. (2008) showed that any solution that satisfies (GCLM ref) must also satisfy
\begin{equation}
\frac{d}{dt}\int_{-\pi}^{\pi}\omega(t,x)\,dx = \int_{-\pi}^{\pi} (-au\omega_x + u_x\omega)\,dx = (a+1)\int_{-\pi}^{\pi}u_x\omega\,dx = (a+1)(H\omega,\omega),
\end{equation}
where  $(\cdot,\cdot)$ denotes the $L^2$ inner product (Okamoto et al., 2008). Therefore, we may assume without loss of generality that the initial condition satisfies
$$\int_{-\pi}^{\pi} \omega(0,x)\,dx = 0.$$

\section{Results}
these are the results

\section{Conclusion}
this is the conclusion

\section{Appendix}
\subsection{Hilbert Transform}
The Hilbert transform is defined as
$$H\w(x,t) = \frac{1}{2\pi}\int_{-\pi}^{\pi}\w(y,t)\cot\left(\frac{x-y}{2}\right)\,dy.$$
The solution $\w(t,y)$ is represented as the truncated Fourier series
$$\w(t,y) = \sum_{k=-N/2}^{N/2 - 1}\w_k(t)\exp(iky).$$
Substituting this into the Hilbert transform
$$H\w(x,t) = \frac{1}{2\pi}\int_{-\pi}^{\pi}\sum_{k=-N/2}^{N/2 - 1}\w_k(t)\exp(iky)\cot\left(\frac{x-y}{2}\right)\,dy.$$
We can swap the integral and sum to obtain
$$H\w(x,t) = \frac{1}{2\pi}\sum_{k=-N/2}^{N/2 - 1}\w_k(t) \int_{-\pi}^{\pi} \exp(iky)\cot\left(\frac{x-y}{2}\right)\,dy.$$
We now apply a substitution $u = y - x$. Therefore,
$$H\w(x,t) = -\frac{1}{2\pi}\sum_{k=-N/2}^{N/2 - 1}\w_k(t) \int_{-\pi-x}^{\pi-x} \exp(ik(u+x))\cot\left(\frac{u}{2}\right)\,du.$$

$$H\w(x,t) = -\frac{1}{2\pi}\sum_{k=-N/2}^{N/2 - 1}\w_k(t)\exp(ikx) \int_{-\pi-x}^{\pi-x} \exp(iku)\cot\left(\frac{u}{2}\right)\,du.$$
Because the integrand is periodic we can shift the domain by $x$
$$H\w(x,t) = -\frac{1}{2\pi}\sum_{k=-N/2}^{N/2 - 1}\w_k(t)\exp(ikx) \int_{-\pi}^{\pi} \exp(iku)\cot\left(\frac{u}{2}\right)\,du.$$
Converting the exponential to its trigonometric form
$$H\w(x,t) = -\frac{1}{2\pi}\sum_{k=-N/2}^{N/2 - 1}\w_k(t)\exp(ikx) \int_{-\pi}^{\pi} [\cos(ku)+i\sin(ku)]\cot\left(\frac{u}{2}\right)\,du,$$
$$H\w(x,t) = -\frac{1}{2\pi}\sum_{k=-N/2}^{N/2 - 1}\w_k(t)\exp(ikx)\left[ \int_{-\pi}^{\pi} \cos(ku)\cot(\frac{u}{2})du +i \int_{-\pi}^{\pi} \sin(ku)\cot\left(\frac{u}{2}\right)\,du\right].$$
The first integral is odd and so is equal to zero. The second integral is even and so can be simplified to a half integral. Therefore,
$$H\w(x,t) = -\frac{2i}{2\pi}\sum_{k=-N/2}^{N/2 - 1}\w_k(t)\exp(ikx)\left[ \int_{0}^{\pi} \sin(ku)\cot\left(\frac{u}{2}\right)\,du\right].$$
The remaining integral is a known identity [CITATION]
$$ \int_{0}^{\pi} \sin(ku)\cot(\frac{u}{2})\,du = \pi \sgn(k), $$
where $\sgn(k)$ is the signum function. Substituting in to the Hilbert transform
$$H\w(x,t) = -i\sum_{k=-N/2}^{N/2 - 1}\w_k(t)\exp(ikx)\sgn(k).$$
\subsection{Hilbert Transform Implementation}

The Hilbert Transform is implemented in Matlab using the code shown below. The Fast Fourier Transform in Matlab formats the Fourier coefficients in order 0,...,N/2 -1 then -N/2,...,-1. 
\\
\begin{lstlisting}[
  style=Matlab-editor,
  language=Matlab
]
% Hilbert Transform Physical
function h = ht(w_t, N)
    % convert w_t to Fourier space
    w_c = fft(w_t,N);
    % wave numbers in fft format
    k = [0:N/2-1, -N/2:-1];
    % signum of wavenumbers
    sgn_k = sign(k); 
    % multiply Fourier coefficients by -i sgn_k
    w_c = (-1i*sgn_k).*w_c;
    % convert w_c back to physical space
    h = ifft(w_c,N);
end
\end{lstlisting}


\subsection{Velocity Field}

The velocity field is defined as
$$u(t,x) = \frac{1}{\pi}\int_{-\pi}^{\pi}\w(t,y)\log\left| \sin\left(\frac{x-y}{2}\right) \right|\,dy.$$
The solution $\w(t,y)$ is represented as the truncated Fourier series
$$\w(t,y) = \sum_{k=-N/2}^{N/2 - 1}\w_k(t)\exp(iky).$$
Substituting this into the velocity field
$$u(t,x) = \frac{1}{\pi}\int_{-\pi}^{\pi} \sum_{k=-N/2}^{N/2 - 1}\w_k(t)\exp(iky)\log\left| \sin\left(\frac{x-y}{2}\right) \right|\,dy.$$
We can swap the integral and sum to obtain
$$u(t,x) = \frac{1}{\pi} \sum_{k=-N/2}^{N/2 - 1}\w_k(t) \int_{-\pi}^{\pi}\exp(iky)\log\left| \sin\left(\frac{x-y}{2}\right) \right|\,dy.$$
We now apply a substitution $v = y - x$. Therefore,
$$u(t,x) = -\frac{1}{\pi} \sum_{k=-N/2}^{N/2 - 1}\w_k(t) \int_{-\pi-x}^{\pi-x}\exp(ik(v+x))\log\left| \sin\left(\frac{v}{2}\right) \right|\,dv.$$
Because the integrand is periodic we can shift the domain by $x$
$$u(t,x) = -\frac{1}{\pi} \sum_{k=-N/2}^{N/2 - 1}\w_k(t) \exp(ikx)\int_{-\pi}^{\pi}\exp(ikv)\log\left| \sin\left(\frac{v}{2}\right) \right|\,dv.$$
Converting the exponential to its trigonometric form
$$u(t,x) = -\frac{1}{\pi} \sum_{k=-N/2}^{N/2 - 1}\w_k(t) \exp(ikx)\int_{-\pi}^{\pi}(\cos(kv) + i\sin(kv))\log\left| \sin\left(\frac{v}{2}\right) \right|\,dv.$$
$$u(t,x) = -\frac{1}{\pi} \sum_{k=-N/2}^{N/2 - 1}\w_k(t) \exp(ikx)\left[\int_{-\pi}^{\pi}\cos(kv)\log\left| \sin\left(\frac{v}{2}\right) \right|\,dv + i\int_{-\pi}^{\pi}\sin(kv)\log\left| \sin\left(\frac{v}{2}\right) \right|\,dv\right].$$
The second integral is odd and so is equal to zero. The first integral is even and so can be simplified to a half integral. Therefore,
$$u(t,x) = -\frac{2}{\pi} \sum_{k=-N/2}^{N/2 - 1}\w_k(t) \exp(ikx)\int_{0}^{\pi}\cos(kv)\log\left| \sin\left(\frac{v}{2}\right) \right|\,dv.$$
The remaining integral is a known identity [CITATION]
$$\int_{0}^{\pi}\cos(kv)\log\left| \sin\left(\frac{v}{2}\right) \right|\,dv = -2\pi\ln(2), \qquad k = 0,$$
$$\int_{0}^{\pi}\cos(kv)\log\left| \sin\left(\frac{v}{2}\right) \right|\,dv = -\frac{\pi}{2|k|}, \qquad k \neq 0.$$
Substituting in to the velocity field
$$u(t,x) = 4\ln(2)w_0(t) + \sum_{k=-N/2,k\neq 0}^{N/2 - 1}\frac{\w_k(t) \exp(ikx)}{k}\,dv.$$

\subsection{Velocity Implementation}

\begin{lstlisting}[
  style=Matlab-editor,
  language=Matlab
]
% Calculate velocity from vorticity
function u = calc_u(w_t,N)
    % convert w_t to Fourier space
    w_c = fft(w_t,N);
    % wave numbers in fft format
    k = [0:N/2-1, -N/2:-1];
    % exclude k=0 
    points = (k ~= 0);
    % calc k=0 val
    w_c(~points) = 4*log(2)*w_c(~points);
    % divide Fourier coefficients by k~=0
    w_c(points) = -w_c(points)./abs(k(points));
    % convert w_c back to physical space
    u = real(ifft(w_c, N));
end
\end{lstlisting}

\pagebreak

\bibliographystyle{plain}
\raggedright
\bibliography{references}

\end{document}
